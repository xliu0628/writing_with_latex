1. Adding accent

With pdfLaTeX, Save your file as UTF-8 and put
\usepackage[utf8]{inputenc}
\usepackage[T1]{fontenc}
in your preamble.
Then you can just type the characters normally into your source file.
e.g. La Ni\~na and El Ni\~no (Right, that's why I will need this trick : )).

2. combine two different-size figures
\begin{figure}
\begin{subfigure}{0.945\textwidth}
\includegraphics[width=0.75\linewidth, height=5cm]{plot_prec_OBS.eps} 
\end{subfigure}
\begin{subfigure}{0.9\textwidth}
\includegraphics[width=0.9\linewidth, height=10cm]{plot_T_U_MSF_OBS.eps}
%\caption{Caption 2}
\end{subfigure}
\caption{The zonally symmetric component of the ENSO signature. (a) Precipitation ($\rm{mm~day^{-1}}$) differences of El Ni\~no minus La Ni\~na. (b) Temperature (colored shading; units: $\rm{^{\circ} C}$) and zonal wind (black contours; units: $\rm{m~s^{-1}}$). The contour interval is for zonal wind is 0.5 $\rm{m~s^{-1}}$; the dashed contour corresponds to the zero line and solid contours denote positive values. Reproduced from Adames and Wallace (2017) with permission.}
 \end{figure}
The trick is changing the subfigure width and figure width.
